\nwfilename{}\nwbegindocs{0}\nwenddocs{}\nwbegindocs{1}\nwdocspar% ===> this file was generated automatically by noweave --- better not edit it
\section{Introduction}
It is often necessary to extract a specific sequence from a file
containing multiple sequences. The program \texttt{getSeq} takes as
input a file containing one or more sequences in FASTA format and a
regular expression. It returns all sequences whose header matches the
regex. Alternatively, it returns all sequences that don't match.
\section{Implementation}
The program consists of header includes, a function to scan one input
file, and the \texttt{main} function.
\nwenddocs{}\nwbegincode{2}\sublabel{NW0-n7igJ-1}\nwmargintag{{\nwtagstyle{}\subpageref{NW0-n7igJ-1}}}\moddef{getSeq.c~{\nwtagstyle{}\subpageref{NW0-n7igJ-1}}}\endmoddef\nwstartdeflinemarkup\nwenddeflinemarkup
\LA{}Includes~{\nwtagstyle{}\subpageref{NW0-ZKEBO-1}}\RA{}
\LA{}Scan file~{\nwtagstyle{}\subpageref{NW0-33zymD-1}}\RA{}
\LA{}Main function~{\nwtagstyle{}\subpageref{NW0-32ejEQ-1}}\RA{}
\nwnotused{getSeq.c}\nwendcode{}\nwbegindocs{3}\nwdocspar
The \texttt{main} function communicates with the user, compiles the
regular expression, and iterates across the input files. It then frees
any memory allocated along the way and finally returns 0 for success.
\nwenddocs{}\nwbegincode{4}\sublabel{NW0-32ejEQ-1}\nwmargintag{{\nwtagstyle{}\subpageref{NW0-32ejEQ-1}}}\moddef{Main function~{\nwtagstyle{}\subpageref{NW0-32ejEQ-1}}}\endmoddef\nwstartdeflinemarkup\nwusesondefline{\\{NW0-n7igJ-1}}\nwenddeflinemarkup
int \nwlinkedidentc{main}{NW0-32ejEQ-1}(int argc, char **argv) \{\nwindexdefn{\nwixident{main}}{main}{NW0-32ejEQ-1}
  \LA{}Communicate with the user~{\nwtagstyle{}\subpageref{NW0-3amXCM-1}}\RA{}
  \LA{}Compile regular expression~{\nwtagstyle{}\subpageref{NW0-20j1hD-1}}\RA{}
  \LA{}Deal with input files~{\nwtagstyle{}\subpageref{NW0-4EgRTZ-1}}\RA{}
  \LA{}Free memory~{\nwtagstyle{}\subpageref{NW0-Dn4oa-1}}\RA{}
  return 0;
\}
\nwused{\\{NW0-n7igJ-1}}\nwidentdefs{\\{{\nwixident{main}}{main}}}\nwendcode{}\nwbegindocs{5}\nwdocspar
The function \texttt{setprogname} is defined in the BSD standard
library.
\nwenddocs{}\nwbegincode{6}\sublabel{NW0-ZKEBO-1}\nwmargintag{{\nwtagstyle{}\subpageref{NW0-ZKEBO-1}}}\moddef{Includes~{\nwtagstyle{}\subpageref{NW0-ZKEBO-1}}}\endmoddef\nwstartdeflinemarkup\nwusesondefline{\\{NW0-n7igJ-1}}\nwprevnextdefs{\relax}{NW0-ZKEBO-2}\nwenddeflinemarkup
#include <bsd/stdlib.h>
\nwalsodefined{\\{NW0-ZKEBO-2}\\{NW0-ZKEBO-3}\\{NW0-ZKEBO-4}\\{NW0-ZKEBO-5}}\nwused{\\{NW0-n7igJ-1}}\nwendcode{}\nwbegindocs{7}\nwdocspar
When communicating with the user, the program may need to identify
itself, so its name is set for later retrieval. Then comes the actual
communication part, where a user might have requested information
about the program or help; (s)he might even have made an error.
\nwenddocs{}\nwbegincode{8}\sublabel{NW0-3amXCM-1}\nwmargintag{{\nwtagstyle{}\subpageref{NW0-3amXCM-1}}}\moddef{Communicate with the user~{\nwtagstyle{}\subpageref{NW0-3amXCM-1}}}\endmoddef\nwstartdeflinemarkup\nwusesondefline{\\{NW0-32ejEQ-1}}\nwenddeflinemarkup
setprogname(argv[0]);
Args *\nwlinkedidentc{args}{NW0-3amXCM-1} = getArgs(argc, argv);\nwindexdefn{\nwixident{args}}{args}{NW0-3amXCM-1}
if(\nwlinkedidentc{args}{NW0-3amXCM-1}->v)
  printSplash(\nwlinkedidentc{args}{NW0-3amXCM-1});
if(\nwlinkedidentc{args}{NW0-3amXCM-1}->h || \nwlinkedidentc{args}{NW0-3amXCM-1}->err)
  printUsage();
\nwused{\\{NW0-32ejEQ-1}}\nwidentdefs{\\{{\nwixident{args}}{args}}}\nwendcode{}\nwbegindocs{9}\nwdocspar
The data structure \texttt{Args} holds the arguments set by the
user. It is defined in \texttt{interface.h}, together with the
functions \texttt{getArgs}, \texttt{printSplash}, and
\texttt{printUsage}.
\nwenddocs{}\nwbegincode{10}\sublabel{NW0-ZKEBO-2}\nwmargintag{{\nwtagstyle{}\subpageref{NW0-ZKEBO-2}}}\moddef{Includes~{\nwtagstyle{}\subpageref{NW0-ZKEBO-1}}}\plusendmoddef\nwstartdeflinemarkup\nwusesondefline{\\{NW0-n7igJ-1}}\nwprevnextdefs{NW0-ZKEBO-1}{NW0-ZKEBO-3}\nwenddeflinemarkup
#include "interface.h"
\nwused{\\{NW0-n7igJ-1}}\nwendcode{}\nwbegindocs{11}\nwdocspar
The argument container is eventually freed.
\nwenddocs{}\nwbegincode{12}\sublabel{NW0-Dn4oa-1}\nwmargintag{{\nwtagstyle{}\subpageref{NW0-Dn4oa-1}}}\moddef{Free memory~{\nwtagstyle{}\subpageref{NW0-Dn4oa-1}}}\endmoddef\nwstartdeflinemarkup\nwusesondefline{\\{NW0-32ejEQ-1}}\nwprevnextdefs{\relax}{NW0-Dn4oa-2}\nwenddeflinemarkup
freeArgs(\nwlinkedidentc{args}{NW0-3amXCM-1});
\nwalsodefined{\\{NW0-Dn4oa-2}}\nwused{\\{NW0-32ejEQ-1}}\nwidentuses{\\{{\nwixident{args}}{args}}}\nwindexuse{\nwixident{args}}{args}{NW0-Dn4oa-1}\nwendcode{}\nwbegindocs{13}\nwdocspar
For the regular expression, space is needed first. The actual
expression is passed via the argument container and is compiled with
the extended regular expression syntax. The compilation might fail, in
which case the program aborts.
\nwenddocs{}\nwbegincode{14}\sublabel{NW0-20j1hD-1}\nwmargintag{{\nwtagstyle{}\subpageref{NW0-20j1hD-1}}}\moddef{Compile regular expression~{\nwtagstyle{}\subpageref{NW0-20j1hD-1}}}\endmoddef\nwstartdeflinemarkup\nwusesondefline{\\{NW0-32ejEQ-1}}\nwenddeflinemarkup
regex_t *re;
re = (regex_t *)emalloc(sizeof(regex_t));
if (regcomp(re, \nwlinkedidentc{args}{NW0-3amXCM-1}->s, REG_EXTENDED) != 0) \{
  fprintf(stderr, "Error[%s] in regex: %s\\n",
            getprogname(), \nwlinkedidentc{args}{NW0-3amXCM-1}->s);
  exit(EXIT_FAILURE);
\}
\nwused{\\{NW0-32ejEQ-1}}\nwidentuses{\\{{\nwixident{args}}{args}}}\nwindexuse{\nwixident{args}}{args}{NW0-20j1hD-1}\nwendcode{}\nwbegindocs{15}\nwdocspar
The function \texttt{emalloc} is defined in the header
\texttt{eprintf.h}.
\nwenddocs{}\nwbegincode{16}\sublabel{NW0-ZKEBO-3}\nwmargintag{{\nwtagstyle{}\subpageref{NW0-ZKEBO-3}}}\moddef{Includes~{\nwtagstyle{}\subpageref{NW0-ZKEBO-1}}}\plusendmoddef\nwstartdeflinemarkup\nwusesondefline{\\{NW0-n7igJ-1}}\nwprevnextdefs{NW0-ZKEBO-2}{NW0-ZKEBO-4}\nwenddeflinemarkup
#include "eprintf.h"
\nwused{\\{NW0-n7igJ-1}}\nwendcode{}\nwbegindocs{17}\nwdocspar
Usage of regular expressions depends on two headers,
\texttt{sys/types.h} and \texttt{regex.h}.
\nwenddocs{}\nwbegincode{18}\sublabel{NW0-ZKEBO-4}\nwmargintag{{\nwtagstyle{}\subpageref{NW0-ZKEBO-4}}}\moddef{Includes~{\nwtagstyle{}\subpageref{NW0-ZKEBO-1}}}\plusendmoddef\nwstartdeflinemarkup\nwusesondefline{\\{NW0-n7igJ-1}}\nwprevnextdefs{NW0-ZKEBO-3}{NW0-ZKEBO-5}\nwenddeflinemarkup
#include <sys/types.h>
#include <regex.h>
\nwused{\\{NW0-n7igJ-1}}\nwendcode{}\nwbegindocs{19}\nwdocspar
The regular expression is freed after use.
\nwenddocs{}\nwbegincode{20}\sublabel{NW0-Dn4oa-2}\nwmargintag{{\nwtagstyle{}\subpageref{NW0-Dn4oa-2}}}\moddef{Free memory~{\nwtagstyle{}\subpageref{NW0-Dn4oa-1}}}\plusendmoddef\nwstartdeflinemarkup\nwusesondefline{\\{NW0-32ejEQ-1}}\nwprevnextdefs{NW0-Dn4oa-1}{\relax}\nwenddeflinemarkup
regfree(re);
\nwused{\\{NW0-32ejEQ-1}}\nwendcode{}\nwbegindocs{21}\nwdocspar
Each input file is searched in turn.
\nwenddocs{}\nwbegincode{22}\sublabel{NW0-4EgRTZ-1}\nwmargintag{{\nwtagstyle{}\subpageref{NW0-4EgRTZ-1}}}\moddef{Deal with input files~{\nwtagstyle{}\subpageref{NW0-4EgRTZ-1}}}\endmoddef\nwstartdeflinemarkup\nwusesondefline{\\{NW0-32ejEQ-1}}\nwenddeflinemarkup
FILE *\nwlinkedidentc{fp}{NW0-4EgRTZ-1};\nwindexdefn{\nwixident{fp}}{fp}{NW0-4EgRTZ-1}
if (\nwlinkedidentc{args}{NW0-3amXCM-1}->nf == 0) \{
  \nwlinkedidentc{fp}{NW0-4EgRTZ-1} = stdin;
  \nwlinkedidentc{scanFile}{NW0-33zymD-1}(\nwlinkedidentc{fp}{NW0-4EgRTZ-1}, \nwlinkedidentc{args}{NW0-3amXCM-1}, re);
\} else \{
  for (int i = 0; i < \nwlinkedidentc{args}{NW0-3amXCM-1}->nf; i++) \{
    \nwlinkedidentc{fp}{NW0-4EgRTZ-1} = efopen(\nwlinkedidentc{args}{NW0-3amXCM-1}->fi[i], "r");
    \nwlinkedidentc{scanFile}{NW0-33zymD-1}(\nwlinkedidentc{fp}{NW0-4EgRTZ-1}, \nwlinkedidentc{args}{NW0-3amXCM-1}, re);
    fclose(\nwlinkedidentc{fp}{NW0-4EgRTZ-1});
  \}
\}
\nwused{\\{NW0-32ejEQ-1}}\nwidentdefs{\\{{\nwixident{fp}}{fp}}}\nwidentuses{\\{{\nwixident{args}}{args}}\\{{\nwixident{scanFile}}{scanFile}}}\nwindexuse{\nwixident{args}}{args}{NW0-4EgRTZ-1}\nwindexuse{\nwixident{scanFile}}{scanFile}{NW0-4EgRTZ-1}\nwendcode{}\nwbegindocs{23}\nwdocspar
Files are defined in the standard library.
\nwenddocs{}\nwbegincode{24}\sublabel{NW0-ZKEBO-5}\nwmargintag{{\nwtagstyle{}\subpageref{NW0-ZKEBO-5}}}\moddef{Includes~{\nwtagstyle{}\subpageref{NW0-ZKEBO-1}}}\plusendmoddef\nwstartdeflinemarkup\nwusesondefline{\\{NW0-n7igJ-1}}\nwprevnextdefs{NW0-ZKEBO-4}{\relax}\nwenddeflinemarkup
#include <stdio.h>
\nwused{\\{NW0-n7igJ-1}}\nwendcode{}\nwbegindocs{25}\nwdocspar
    The function \texttt{scanFile} iterates across the lines in a
    file. When it encounters a sequence header, it decides whether the
    corresponding sequence is printed. 
\nwenddocs{}\nwbegincode{26}\sublabel{NW0-33zymD-1}\nwmargintag{{\nwtagstyle{}\subpageref{NW0-33zymD-1}}}\moddef{Scan file~{\nwtagstyle{}\subpageref{NW0-33zymD-1}}}\endmoddef\nwstartdeflinemarkup\nwusesondefline{\\{NW0-n7igJ-1}}\nwenddeflinemarkup
void \nwlinkedidentc{scanFile}{NW0-33zymD-1}(FILE *\nwlinkedidentc{fp}{NW0-4EgRTZ-1}, Args *\nwlinkedidentc{args}{NW0-3amXCM-1}, regex_t *re) \{\nwindexdefn{\nwixident{scanFile}}{scanFile}{NW0-33zymD-1}
  static char *line = NULL;
  size_t len = 0;
  int match = 0;
   while (getline(&line, &len, \nwlinkedidentc{fp}{NW0-4EgRTZ-1}) != -1) \{
    if (line[0] == '>') \{
        \LA{}Decide about match~{\nwtagstyle{}\subpageref{NW0-2Y1GqX-1}}\RA{}
    \}
    \LA{}Respond to match~{\nwtagstyle{}\subpageref{NW0-3Xa1TF-1}}\RA{}
   \}
\}
\nwused{\\{NW0-n7igJ-1}}\nwidentdefs{\\{{\nwixident{scanFile}}{scanFile}}}\nwidentuses{\\{{\nwixident{args}}{args}}\\{{\nwixident{fp}}{fp}}}\nwindexuse{\nwixident{args}}{args}{NW0-33zymD-1}\nwindexuse{\nwixident{fp}}{fp}{NW0-33zymD-1}\nwendcode{}\nwbegindocs{27}\nwdocspar
    Decide whether or not a header line is a match by using a function of
    the \texttt{regex} library.
\nwenddocs{}\nwbegincode{28}\sublabel{NW0-2Y1GqX-1}\nwmargintag{{\nwtagstyle{}\subpageref{NW0-2Y1GqX-1}}}\moddef{Decide about match~{\nwtagstyle{}\subpageref{NW0-2Y1GqX-1}}}\endmoddef\nwstartdeflinemarkup\nwusesondefline{\\{NW0-33zymD-1}}\nwenddeflinemarkup
if (regexec(re, line, 0, NULL, 0) == 0)
  match = 1;
else
  match = 0;
\nwused{\\{NW0-33zymD-1}}\nwendcode{}\nwbegindocs{29}\nwdocspar
A matching sequence gets printed if the \emph{complement} option is
off, and a non-matching sequence gets printed if the complement option
is on.
\nwenddocs{}\nwbegincode{30}\sublabel{NW0-3Xa1TF-1}\nwmargintag{{\nwtagstyle{}\subpageref{NW0-3Xa1TF-1}}}\moddef{Respond to match~{\nwtagstyle{}\subpageref{NW0-3Xa1TF-1}}}\endmoddef\nwstartdeflinemarkup\nwusesondefline{\\{NW0-33zymD-1}}\nwenddeflinemarkup
if ((match && !\nwlinkedidentc{args}{NW0-3amXCM-1}->c) || (!match && \nwlinkedidentc{args}{NW0-3amXCM-1}->c))
  printf("%s", line);
\nwused{\\{NW0-33zymD-1}}\nwidentuses{\\{{\nwixident{args}}{args}}}\nwindexuse{\nwixident{args}}{args}{NW0-3Xa1TF-1}\nwendcode{}\nwbegindocs{31}\nwdocspar
This concludes the implementation of \texttt{getSeq}.
\nwenddocs{}\nwbegindocs{32}\nwdocspar
\section{List of code chunks}
\nowebchunks
\section{Index}
\nowebindex
\nwenddocs{}

\nwixlogsorted{c}{{Communicate with the user}{NW0-3amXCM-1}{\nwixu{NW0-32ejEQ-1}\nwixd{NW0-3amXCM-1}}}%
\nwixlogsorted{c}{{Compile regular expression}{NW0-20j1hD-1}{\nwixu{NW0-32ejEQ-1}\nwixd{NW0-20j1hD-1}}}%
\nwixlogsorted{c}{{Deal with input files}{NW0-4EgRTZ-1}{\nwixu{NW0-32ejEQ-1}\nwixd{NW0-4EgRTZ-1}}}%
\nwixlogsorted{c}{{Decide about match}{NW0-2Y1GqX-1}{\nwixu{NW0-33zymD-1}\nwixd{NW0-2Y1GqX-1}}}%
\nwixlogsorted{c}{{Free memory}{NW0-Dn4oa-1}{\nwixu{NW0-32ejEQ-1}\nwixd{NW0-Dn4oa-1}\nwixd{NW0-Dn4oa-2}}}%
\nwixlogsorted{c}{{getSeq.c}{NW0-n7igJ-1}{\nwixd{NW0-n7igJ-1}}}%
\nwixlogsorted{c}{{Includes}{NW0-ZKEBO-1}{\nwixu{NW0-n7igJ-1}\nwixd{NW0-ZKEBO-1}\nwixd{NW0-ZKEBO-2}\nwixd{NW0-ZKEBO-3}\nwixd{NW0-ZKEBO-4}\nwixd{NW0-ZKEBO-5}}}%
\nwixlogsorted{c}{{Main function}{NW0-32ejEQ-1}{\nwixu{NW0-n7igJ-1}\nwixd{NW0-32ejEQ-1}}}%
\nwixlogsorted{c}{{Respond to match}{NW0-3Xa1TF-1}{\nwixu{NW0-33zymD-1}\nwixd{NW0-3Xa1TF-1}}}%
\nwixlogsorted{c}{{Scan file}{NW0-33zymD-1}{\nwixu{NW0-n7igJ-1}\nwixd{NW0-33zymD-1}}}%
\nwixlogsorted{i}{{\nwixident{args}}{args}}%
\nwixlogsorted{i}{{\nwixident{fp}}{fp}}%
\nwixlogsorted{i}{{\nwixident{main}}{main}}%
\nwixlogsorted{i}{{\nwixident{scanFile}}{scanFile}}%

